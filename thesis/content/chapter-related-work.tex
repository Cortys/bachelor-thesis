% !TEX root = ../main.tex
%
\chapter{Verwandte Arbeiten}
\label{sec:related}

Die in~\ref{sec:intro:goals} beschriebenen Ziele werden bereits seit langem erforscht.
Der Begriff \textit{Wissensgraph} wurde 2012 durch Google popularisiert, die Ideen dahinter lassen sich allerdings bis ins Ende des 19. Jahrhunderts zurückverfolgen.
Dieses Kapitel zeigt auf, wie sich die Themen dieser Arbeit in die bisherige Forschung einfügen.
\ref{sec:related:kr} ordnet das Konzept des Wissensgraphen in die Entwicklungsgeschichte der Wissensrepräsentation ein.
\ref{sec:related:kbc} beschreibt die aktuell verwendeten Verfahren zur Konstruktion von Wissensgraphen.
In~\ref{sec:related:nlp} wird schließlich ein Überblick über die momentan verbreiteten NLP (\textit{natural language processing}) Werkzeuge gegeben.

\section{Ansätze zur Wissensrepräsentation}
\label{sec:related:kr}

\subsection{Logische Grundlagen}
\label{sec:related:kr:logic}

\subsubsection{Begriffsschrift (1879)}
\label{sec:related:kr:logic:begriff}

\subsubsection{Existential Graphs (1882)}
\label{sec:related:kr:logic:eg}

\subsubsection{Prädikatenlogik (1908)}
\label{sec:related:kr:logic:pred}

\subsection{Entwicklung maschineller Wissensrepräsentation}
\label{sec:related:kr:history}

\subsubsection{General Problem Solver (1959)}
\label{sec:related:kr:history:gps}

\subsubsection{Expertensysteme (1970)}
\label{sec:related:kr:history:expert}

\subsubsection{Conceptual Graphs (1976)}
\label{sec:related:kr:history:cg}

\subsection{Aktuelle Wissensrepräsentationsansätze}
\label{sec:related:kr:today}

\subsubsection{Semantic Web}
\label{sec:related:kr:today:sw}

\subsubsection{NELL}
\label{sec:related:kr:today:nell}

\subsubsection{Google Knowledge Graph}
\label{sec:related:kr:today:google-kg}

\section{Konstruktionsansätze für Wissensgraphen}
\label{sec:related:kbc}

\section{NLP Werkzeuge}
\label{sec:related:nlp}
