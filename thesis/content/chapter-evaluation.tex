% !TEX root = ../main.tex
% chktex-file 46
\chapter{Bewertung}%
\label{sec:evaluation}

Die Bewertung des im vorigen Kapitel vorgestellten Verfahrens erfolgt in zwei Schritten.
In~\ref{sec:evaluation:quality} wird im ersten Schritt die Qualität der konstruierten Wissensgraphen untersucht.
Anschließend wird in~\ref{sec:evaluation:time} die Tauglichkeit des Verfahrens im Kontext von Nachrichtenstreams betrachtet.

\section{Qualitative Bewertung}%
\label{sec:evaluation:quality}

Das vorgestellte Verfahren macht zahlreiche Annahmen über die Struktur und Komplexität der Eingabedaten, zudem wird lediglich ein kleiner Teil aller potentiell in Annotationsgraphen enthaltenen Informationen genutzt.
In diesem Kapitel wird betrachtet, wie sich diese Einschränkungen auf die Qualität der resultierenden Wissensgraphen auswirken.
Ziel ist dabei nicht, eine umfassende empirische Auswertung durchzuführen, sondern stattdessen die Stärken und Schwächen des Verfahrens anhand von stichprobenartig ausgewählten, realen Beispieldaten aufzuzeigen.
Es gibt im Wesentlichen zwei Gründe für diese Entscheidung:
\begin{enumerate}
	\item Das entwickelte Verfahren ist primär prototypisch zu verstehen.
		Das Ziel war nicht, ein bestimmtes Maß an Qualität zu erreichen, sondern vielmehr die Praktikabilität des Ansatzes zu demonstrieren.
	\item Es existieren keine Testdatensets, die für eine empirische Analyse des entwickelten Verfahrens geeignet sind.
		Es existieren zwar Testdaten für die verwendeten NLP-Tasks (NER, POS-Tagging, Coreference Resolution etc.), da die Ergebnisse hierfür allerdings direkt aus CoreNLP stammen, würde mit den verfügbaren Testdaten lediglich die Performance von CoreNLP gemessen.
		Benötigt wären Testdaten, die das Evaluieren der Konzept\-graph\-struktur erlauben, d.~h.\ solche, die Informationen über die semantischen Abhängigkeiten und Negation von Konzepten enthalten.
		Fakten-Benchmarks wie \textit{FactBench}~\cite{FactBench} wären grundsätzlich für die Performance-Bewertung geeignet, hierfür wäre allerdings ein Set von Kommunikationsdaten notwendig, welches Informationen über die im Benchmark geprüften Fakten enthält;
		ein solches Datenset wurde bei der Recherche im Rahmen dieser Arbeit nicht gefunden.
\end{enumerate}

\subsection{Bewertungsansatz}%
\label{sec:evaluation:quality:method}

Da für die Bewertung nicht auf Referenzergebnisse zurückgegriffen werden kann, müssen die Ergebnisse manuell ausgewertet werden.
Das manuelle Auswerten einer repräsentativen Stichprobe von Testnachrichten hätte den zeitlichen Rahmen dieser Arbeit gesprengt.
Daher wird stattdessen mit einer kleinen Menge von Testnachrichten gearbeitet.
Um das Ergebnis dieser Nachrichten zu bewerten, wird der aus ihnen konstruierte Wissensgraph im ersten Schritt in eine Neo4j-Graphdatenbank~\cite{Neo4j} eingefügt.
Anschließend wird daran eine Menge von Testanfragen~\tref{sec:appendix:queries} gestellt und die Ergebnisse manuell mit den in den Nachrichtentexten enthaltenen Informationen abgeglichen.
Auf diesem Wege kann zwar die Wissensgraphqualität nicht akkurat beurteilt werden, es ist allerdings möglich tendenzielle Stärken und Schwächen des Verfahrens zu erkennen.

\subsection{Testdaten}%
\label{sec:evaluation:quality:data}

Die verwendeten Textnachrichten stammen aus dem Enron E-Mail Datenset~\cite{Cohen2015}.
Da das zufällige Auswählen von E-Mails aus diesem Datenset i.~d.~R. in unzusammenhängenden Nachrichtensets resultiert, wurden die Enron Daten nicht direkt benutzt, sondern stattdessen der Enron Threads Corpus~\cite{Jamison2013}\cite{EnronThreads}.
Hierbei handelt es sich um ein Datenset, welches die E-Mail-Rohdaten in zusammengehörige Kommunikationsthreads gruppiert.
Die Nachrichten innerhalb eines Threads sind gut als Testdaten geeignet, da darin oftmals dieselben Personen und Themen auftauchen und somit die Möglichkeit besteht, Beziehungen zwischen diesen zu finden.

Für die Testnachrichten wurde zufällig ein Testthread aus der Menge aller Threads ausgewählt, der die Schlüsselwörter \textit{``yesterday''} oder \textit{``tomorrow''} enthält.
Ziel dabei war es, Nachrichten, in denen es um Termine oder Ereignisse geht, zu erhalten, da diese Nachrichten häufig auch ohne umfangreiches domänenspezifisches Wissen verstanden werden können.
Die Wahl eines zufälligen Threads, ohne diese Einschränkung, resultierte meist in Nachrichten, die nur mit umfangreichem domänenspezifischem Wissen aus der Energie-Branche verständlich sind.
Die konkret verwendeten Testnachrichten sind in~\ref{sec:appendix:msgs} zu finden.

\subsection{Ergebnisse}%
\label{sec:evaluation:quality:results}

In diesem Abschnitt wird der aus den Testdaten konstruierte Wissensgraph\footnote{Es befindet sich ein Neo4j-Datenbank-Export des Wissensgraphen unter \url{https://github.com/Cortys/bachelor-thesis/blob/master/thesis/data/evaluation/testKG.cql}, mit diesem können die Ergebnisse nachvollzogen werden.} untersucht.
Hierfür werden vier Testanfragen benutzt.
Die Ergebnisse dieser Anfragen werden auf Korrektheit geprüft und benutzt, um Verbesserungspotential des Verfahrens aufzuzeigen.
Dabei wird zwischen zwei Arten von Fehlern unterschieden:
\newcommand{\errorA}{{\color{rot} (A)}}
\newcommand{\errorB}{{\color{blau} (B)}}
\begin{tasks}[label-offset=1em](2) % chktex 36
	\task[\errorA] \color{rot}Fehlerhafte Informationen
	\task[\errorB] \color{blau}Fehlende Informationen
\end{tasks}

\paragraph{Personen}
Die erste Anfrage~\tref{sec:appendix:queries:people} testet, welche Personen in den Testnachrichten erkannt wurden.
Da dieselbe Person in verschiedenen Kontexten verschiedene Namen haben kann, ist die Ausgabe der Anfrage eine Liste von Personen, die jeweils durch eine Liste von Namen beschrieben werden.
\csvautobooklongtable[separator=semicolon]{data/evaluation/people.csv}
In den Testnachrichten werden fünf Personen erkannt.
Dabei lassen sich zwei Fehler beobachten:
\begin{enumerate}
	\item Es kommen nur vier Personen in den Nachrichten vor.
		Der Bezeichner \textit{Dr. Lu} von Person~5 wurde nicht als koreferent zu den Bezeichnern von Person~1 erkannt. % chktex 12
		Dies liegt daran, dass für die Erkennung von koreferenten Namen lediglich die CoreNLP Koreferenzinformationen und die Jaro-Distanzen von $named$-Konzepten berücksichtigt werden.
		Das Ergebnis ließe sich verbessern, indem weitere Informationen, wie z.~B. der Kontext, in dem ein Bezeichner auftaucht berücksichtigt werden. \errorB\
	\item Person~1 hat die drei fehlerhaften Bezeichner \textit{TX Zimin}, \textit{address} und \textit{Zimin Lu Enron Corp} erhalten.
		Diese Fehler stammen aus dem Dependency Parser von CoreNLP, sie lassen sich mit der aktuellen Architektur, in der alle Informationen über den Eingabetext aus den CoreNLP-Annotatoren stammen, also nicht vermeiden.
		Die Ursache des Fehlers ist die Testnachricht~6~\tref{sec:appendix:msgs}, in der eine Postanschrift vorkommt;
		damit kann CoreNLP in der verwendeten Version~3.8.0 noch nicht umgehen. \errorA\
\end{enumerate}

\paragraph{Ereignisse und Termine}
Die zweite Anfrage~\tref{sec:appendix:queries:events} testet, wie gut $relation$-Konzepte erkannt werden.
Hierfür wird aufgelistet, was \textit{Duane Seppi} wann getan hat.
\csvautobooklongtable{data/evaluation/personTimeAction.csv}
Wie zu erwarten, finden sich im Ergebnis für alle gesendeten und empfangenen Nachrichten entsprechende \textit{send}- bzw. \textit{receive}-Einträge.
Unter den übrigen Einträgen, wurden den ersten beiden Einträgen Zeitpunkte zugeordnet.
Beide stammen aus der ersten Testnachricht und sind im Wesentlichen korrekt, die relativen Zeitangaben \textit{yesterday} und \textit{today} wurden von SUTime korrekt aufgelöst und die PSL-Regel \ruleref{rule:text2kg:i} hat die Pronomina, die Agens der $relation$-Konzepte sind, korrekt der Person \textit{Duane Seppi} zugeordnet.
Es wurden allerdings auch verschiedene Fehler gemacht:
\begin{enumerate}
	\item In den Einträgen 8 und 9 spiegelt sich wider, dass Fehler im CoreNLP Abhängigkeitsgraphen eine semantisch inkorrekte Repräsentation des ersten Satzes von Nachricht~3 verursacht haben:
	 	\textit{``I would be grateful if I could talk with you some time about the typical terms one sees in Swing, take or pay, virtual storage, etc.\ options.''} \errorA\
	\item Die Einträge 15 und 16 sind fehlerhaft, da bei der CoreNLP Coreference Resolution das Pronomen \textit{I} des zweiten Satzes von Nachricht~6 fälschlicherweise mit \textit{Duane Seppi} verknüpft wurde. \errorA\
	\item Die Aussage von Satz~2 aus Nachricht~3 wird nicht im Wissensgraphen repräsentiert:
		\textit{``This is related to some research some colleagues and I are doing applying recent innovations in Monte Carlo valuation of options with early exercise.''} \errorB\

		Der Grund hierfür ist, dass nicht alle Universal Dependencies Abhängigkeits\-typen auf ihre semantische Entsprechung in der Wissensgraphontologie abgebildet werden.
		Der obige Satz wird z.~B. deshalb nicht korrekt im Graphen repräsentiert, da die Abhängigkeitstypen \texttt{advcl} und \texttt{acl} bislang nicht berücksichtigt werden.
\end{enumerate}

\paragraph{Personenbezogene Daten}
Die dritte Anfrage~\tref{sec:appendix:queries:numbers} testet, wie gut Faktenwissen, welches in der Eingabe vorhanden ist, erkannt und aus dem Wissensgraphen ausgelesen werden kann.
In den verwendeten Testnachrichten teilen sich zwei Personen ihre Telefonnummern mit.
Die Anfrage nach allen Nummern und zugehörigen Nummertypen (Telefon-, Fax-, Hausnummer etc.) aller Personen, liefert folgendes Ergebnis:
\csvautobooklongtable[separator=semicolon]{data/evaluation/numbers.csv}
Die Telefonnummern wurden korrekt erkannt.
Bei der ersten Nummer fehlt jedoch die Information, dass es sich um eine Telefonnummer handelt \errorB;\@
dies liegt daran, dass in der Nachricht, in der sie vorkommt nicht explizit erwähnt wird, um was für eine Art Nummer es sich handelt.
Um dies zu erkennen, müssten Kommunikationsverläufe berücksichtigt werden, statt Nachrichten weitestgehend isoliert zu betrachten.

\paragraph{Positive und negative Aussagen}
In den vorigen Anfragen wurden Kontexte nicht berücksichtigt, da die Testnachrichten größtenteils Tatsachen beschreiben.
Die vierte Anfrage~\tref{sec:appendix:queries:neg} wird daher benutzt, um die Erkennung von Negationskontexten zu testen.
Konkret werden hierfür alle Relationen abgefragt, die in, von \textit{John Doe} geschriebenen, Nachrichten erwähnt werden und für die etwas über deren Wahrheit oder Unwahrheit bekannt ist.
\csvautobooklongtable[separator=semicolon]{data/evaluation/personNegationAction.csv}
Relation~2 wurde dabei semantisch inkorrekt abgebildet \errorA.
Die Negation von \textit{``I have something''} im Satz \textit{``I'm not writing this message because I have something to say''} wurde nicht erkannt.
Grund hierfür ist zum einen, dass der Abhängigkeitstyp \texttt{advcl} in der NLP-Phase nicht berücksichtigt wird.
Außerdem fehlt das Hintergrundwissen über die Bedeutung des Wortes \textit{because}.
Um den Satz korrekt zu verstehen, wäre es notwendig die Implikation $\textit{I have something} \rightarrow \lnot(\textit{I write message})$ zu erkennen, mit dem Wissen, dass \textit{I write message} wahr ist, zu verknüpfen und schließlich $\lnot(\textit{I have something})$ zu folgern.
Zu derartige Schlussfolgerungen ist das vorgestellte Verfahren bislang nicht in der Lage.

\section{Laufzeituntersuchung}%
\label{sec:evaluation:time}

In diesem Abschnitt wird untersucht, inwiefern sich das beschriebene Verfahren im Kontext von Nachrichtenstreams eignet.
Da in der PSL-Phase für jede eingefügte Nachricht momentan eine vollständige ADMM-basierte Inferenz durchgeführt wird, ist zu erwarten, dass mit jeder eingefügten Nachricht die Inferenzdauer steigt.
Dies wurde überprüft, indem der Reihe nach die Nachrichten aus drei verschiedenen Enron E-Mail Threads eingefügt wurden;
verwendet wurden die zufällig ausgewählten Threads \texttt{bin1/300249}, \texttt{bin1/363227} und \texttt{bin5/350143}.
Nachrichten ohne Inhalt, From- oder To-Header wurden ignoriert.
Insgesamt wurden 57 Nachrichten der Reihe nach eingefügt%
\footnote{Die Tests wurden auf einem Desktop-Rechner mit einem AMD Ryzen~7~1800X Prozessor (8~Kerne, 16~Threads, 3.6~GHz) und 32~GB RAM unter Linux mit der Kernel-Version 4.13.5 und OpenJDK~8u162 durchgeführt.
Da ein Testdurchlauf knapp 42 Stunden benötigt, wurden aus zeitlichen Gründen lediglich zwei Durchläufe durchgeführt und gemittelt.}.
\begin{figure}[h]
	\centering
	\begin{tikzpicture}
		\pgfplotstableread[col sep=comma] {data/evaluation/perfBig.csv}\evaluationPerfData%
		\begin{axis}[
			width=\textwidth,
			height=6cm,
			axis x line=bottom,
			axis line style={draw=none},
			ytick style={draw=none},
			ymajorgrids,
			ylabel={Inferenzdauer in Sek.}
		]
			\addplot+[only marks,restrict x to domain=1:24] table [x=i, y=time]{\evaluationPerfData}
				node[above=2pt,pos=0.5]{\texttt{bin1/300249}};
			\addplot+[only marks,restrict x to domain=25:32] table [x=i, y=time]{\evaluationPerfData}
				node[above=16pt,pos=0.5]{\texttt{bin1/363227}};
			\addplot+[only marks,restrict x to domain=33:57] table [x=i, y=time]{\evaluationPerfData}
				node[above=15pt,pos=0.5]{\texttt{bin5/350143}};
		\end{axis}
	\end{tikzpicture}
	\caption{Inferenzdauer beim sequentiellen Einfügen der Testnachrichten (Tabelle in~\ref{sec:appendix:perf})}\label{fig:evaluation:perf}
\end{figure}
Wie erwartet steigt die Inferenzdauer mit jeder eingefügten Nachricht tendenziell an; es lässt sich jedoch beobachten, dass das Einfügen der ersten Nachricht aus einem Thread eine deutliche Erhöhung der Inferenzdauer verursacht.
Eine Erklärung hierfür ist, dass die Stärke des Anstiegs der Inferenzdauer, von der Anzahl neuer Konzepte in einer Nachricht abhängt.
Die Inferenzdauer-Anstiege zwischen zwei Threads werden dadurch verursacht, dass die erste Nachricht eines neuen Threads i.~d.~R. viele neue Konzepte einführt.
Diese Erklärung wird durch die Korrelation des Anstiegs der Inferenzdauer $\Delta t_i = t_i - t_{i - 1}$ zwischen zwei Nachrichten $m_{i - 1}, m_i$ und der Anzahl neuer Konzepte $c_i$ in der Nachricht $m_i$ unterstützt.
Die Korrelation $Kor(\Delta t, c)$ über die gesamte Messreihe beträgt lediglich $0.187$, da insbesondere im ersten Thread konstante Faktoren die Laufzeit dominieren.
Betrachtet man daher nur die Nachrichten des zweiten und dritten Threads $m_{25}, \dots, m_{57}$ ergibt sich jedoch $Kor_{25}(\Delta t, c) = 0.616$.

Die gemessene Inferenzdauer pro Nachricht liegt anfangs bei ca.\ 4 Sek.\ und steigt auf ca.\ 2 Std.\ an.
Die momentane Implementation des Verfahrens ist also definitv nicht für den praktischen Einsatz geeignet.
Es existieren jedoch verschiedene Ansätze, mit denen die absoluten und aymptotischen Laufzeiten deutlich verbessert werden können.
Eine triviale Optimierung ist es, nicht alle Nachrichten einzeln, sondern mehrere gleichzeitig einzufügen und so, weniger Inferenzen durchzuführen.
Wenn erst der Konjunktionsgraph aller Nachrichten gebildet und dann eine einzige Inferenz durchgeführt wird, benötigt das Einfügen der Nachrichten ca. 2~Stunden, statt 42~Stunden.
Weitere Ansätze werden in~\ref{sec:conclusion:todo} beschrieben.
