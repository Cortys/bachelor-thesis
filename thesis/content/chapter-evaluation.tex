% !TEX root = ../main.tex
% chktex-file 46
\chapter{Bewertung}%
\label{sec:evaluation}

Die Bewertung des im vorigen Kapitel vorgestellten Verfahrens erfolgt in zwei Schritten.
In~\ref{sec:evaluation:quality} wird im ersten Schritt die Qualität der konstruierten Wissensgraphen untersucht.
Anschließend wird in~\ref{sec:evaluation:time} die Tauglichkeit des Verfahrens im Kontext von Nachrichtenstreams betrachtet.

\section{Qualitative Bewertung}%
\label{sec:evaluation:quality}

Das vorgestellte Verfahren macht zahlreiche Annahmen über die Struktur und Komplexität der Eingabedaten, zudem wird lediglich ein kleiner Teil aller potentiell in Annotationsgraphen enthaltenen Informationen genutzt.
In diesem Kapitel wird betrachtet, wie sich diese Einschränkungen auf die Qualität der resultierenden Wissensgraphen auswirken.
Ziel ist dabei nicht eine umfassende empirische Auswertung durchzuführen, sondern stattdessen die Stärken und Schwächen des Verfahrens anhand von stichprobenartig ausgewählten, realen Beispieldaten aufzuzeigen.
Es gibt im Wesentlichen drei Gründe für diese Entscheidung:
\begin{enumerate}
	\item Das entwickelte Verfahren ist primär prototypisch zu verstehen.
		Das Ziel war nicht ein bestimmtes Maß an Qualität zu erreichen, sondern vielmehr die Praktikabilität des Ansatzes zu demonstrieren.
	\item Es gibt keine alternativen Ansätze mit dem das entwickelte Verfahren verglichen werden kann.
		Eine automatisierte Konzeptgraph-basierte Wissensgraphkonstruktion aus natürlichsprachlichen Nachrichten wurde bislang in keiner anderen Arbeit untersucht.
	\item Es existieren zudem keine Testdatensets, die für eine empirische Analyse des entwickelten Verfahrens geeignet sind.
		Es existieren zwar Testdaten für die verwendeten NLP-Tasks (NER, POS-Tagging, Coreference Resolution etc.), da die Ergebnisse hierfür allerdings direkt aus CoreNLP stammen, würde mit den verfügbaren Testdaten lediglich die Performance von CoreNLP gemessen.
		Benötigt sind Testdaten, die das Evaluieren der Konzept\-graph\-struktur erlauben, d.~h.\ solche, die Informationen über die semantischen Abhängigkeiten und Negation von Konzepten enthalten.
\end{enumerate}

\subsection{Bewertungsansatz}%
\label{sec:evaluation:quality:method}

Da für die Bewertung nicht auf Referenzergebnisse zurückgegriffen werden kann, müssen die Ergebnisse manuell ausgewertet werden.
Das manuelle Auswerten einer repräsentativen Stichprobe von Testnachrichten hätte den zeitlichen Rahmen dieser Arbeit gesprengt.
Daher wird stattdessen mit einer kleinen Menge von Testnachrichten gearbeitet.
Um das Ergebnis dieser Nachrichten zu bewerten, wird der aus ihnen konstruierte Wissensgraph im ersten Schritt in eine Neo4j-Graphdatenbank eingefügt.
Anschließend wird darauf eine Menge von Testanfragen ausgeführt und die Ergebnisse manuell mit den in den Nachrichtentexten enthaltenen Informationen abgeglichen.
Auf diesem Wege kann zwar die Wissensgraphqualität nicht akkurat beurteilt werden, es ist allerdings möglich tendenzielle Stärken und Schwächen des Verfahrens zu erkennen.

\subsection{Testdaten}%
\label{sec:evaluation:quality:data}

Die verwendeten Textnachrichten stammen aus dem Enron E-Mail Datenset~\cite{Cohen2015}.
Da das zufällige Auswählen von E-Mails aus diesem Datenset i.~d.~R. in unzusammenhängenden Nachrichtensets resultiert, wurden die Enron Daten nicht direkt benutzt, sondern stattdessen der Enron Threads Corpus~\cite{Jamison2013}\cite{EnronThreads}.
Hierbei handelt es sich um ein Datenset, welches die E-Mail-Rohdaten in zusammengehörige Kommunikationsthreads gruppiert.
Die Nachrichten innerhalb eines Threads sind gut als Testdaten geeignet, da darin oftmals die selben Personen und Themen auftauchen und somit die Möglichkeit besteht, Beziehungen zwischen diesen zu finden.

Für die Testnachrichten wurde zufällig ein Testthread aus der Menge aller Threads ausgewählt, der die Schlüsselwörter \textit{``yesterday''} oder \textit{``tomorrow''} enthält.
Ziel dabei war es, Nachrichten, in denen es um Termine oder Ereignisse geht, zu erhalten, da diese Nachrichten häufig auch ohne umfangreiches domänenspezifisches Wissen verstanden werden können.
Die Wahl eines beliebigen Threads resultierte meist in Nachrichten, die nur mit domänenspezifischem Wissen aus der Energie-Branche verständlich sind.
Die konkret verwendeten Testnachrichten sind in~\ref{sec:appendix:testmsgs} zu finden.

\subsection{Ergebnisse}%
\label{sec:evaluation:quality:results}



\section{Laufzeituntersuchung}%
\label{sec:evaluation:time}
