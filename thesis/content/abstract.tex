% !TEX root = ../main.tex
%
\pdfbookmark[0]{Abstract}{Abstract}
\chapter*{Abstract}%
\label{sec:abstract}
\vspace*{-10mm}

In dieser Arbeit wird ein Verfahren zur automatisierten Wissensgraph-Konstruktion aus natürlichsprachlichen textuellen Kommunikationsdaten (z.~B. E-Mails) vorgestellt.
Das vorgestellte Verfahren transformiert einen Stream von Textnachrichten sukzessive in einen Wissensgraphen.
Es arbeitet in zwei Schritten:
Im ersten Schritt wird ein Natural Language Processing (NLP) einer eingegebenen Nachricht mit der Stanford CoreNLP Bibliothek durchgeführt.
Das im NLP-Schritt extrahierte Wissen wird anschließend mittels der Probabilistic Soft Logic (PSL) in den vorhandenen Wissensgraphen eingefügt.
Die Struktur des so konstruierten Graphen baut auf J.~F.~Sowas Konzeptgraphen auf.
Neu in dieser Arbeit ist die Kombination von Konzeptgraphen, CoreNLP und PSL.\@
Der Vorteil dieser Kombination ist die hohe Ausdrucksstärke der konstruierten Wissensgraphen und die Skalierbarkeit des Verfahrens.
Das Verfahren wurde implementiert und wird anhand realer Testdaten bewertet.
