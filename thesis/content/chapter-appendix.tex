% !TEX root = ../main.tex
% chktex-file 46

\chapter{Anhang}%
\label{sec:appendix}

\section{Verwendete Testnachrichten}%
\label{sec:appendix:msgs}

Die Testnachrichten 1--7 stammen aus dem Thread \texttt{bin1/363227} des Enron Threads Corpus~\cite{EnronThreads}.
Die Testnachricht 8 wurde nachträglich hinzugefügt, da in den vorherigen Nachrichten keine Negation vorkam.
Da das implementierte Wissensgraphkonstruktionsverfahren nicht speziell für E-Mails ausgelegt ist, wird nicht zwischen Absender und Absender-Adresse unterschieden.
Die Absender- und Empfänger-Adressen in den Rohdaten wurden daher durch die Namen der Absender bzw. Empfänger ersetzt.
Abgesehen von dieser Änderung, wurden die Inhalte und Absendezeitpunkte der Mails unverändert übernommen.
\inputminted{clojure}{data/evaluation/testdata.clj}

\section{Verwendete Testanfragen}%
\label{sec:appendix:queries}

Zur Evaluation des aus den Testnachrichten konstruierten Wissensgraphen wurden drei Testanfragen verwendet, die jeweils verschiedene Aspekte des Graphen betrachten.
Die Anfragen sind in der Sprache Cypher des Neo4j-Graphdatenbanksystems~\cite{Neo4j} geschrieben.
Die Tests wurden mit Neo4j~3.2.6 und dem Graph Algorithms Plugin~3.2.2.1~\cite{GraphAlgo} durchgeführt.

\subsection{Personen}%
\label{sec:appendix:queries:people}

\inputminted{cypher}{data/evaluation/people.cql}
Diese Anfrage arbeitet im Wesentlichen in zwei Schritten.
Im ersten Schritt wird der Teilgraph aller Konzepte gebildet, die Instanz von \textit{person} und nicht Instanz von \textit{I} oder \textit{you} sind.
Im zweiten Schritt wird dann die Menge aller über $inst$-Kanten schwach zusammenhängenden Komponenten des Teilgraphen ermittelt.
Die $label$ der Konzepte innerhalb einer Zusammenhangskomponente sind die Namen bzw.\ Bezeichner, die die selbe Person in verschiedenen Nachrichten hat.

Der Grund für diesen Aufbau ist, dass eine Person als ein Konzept mit gewissen charakterisierenden Eigenschaften verstanden werden kann.
Die Existenz von $inst(A, B)$ zwischen zwei Personen-Konzepten $A, B$ bedeutet, dass $B$ die charakterisierenden Eigenschaften von $A$ besitzt und somit beide Konzepte auf die selbe Person verweisen.
Für die Modellierung dieser Idee werden keine Koreferenzkanten verwendet, da zwei Konzepte, die auf die selbe Person verweisen, dennoch verschieden sein können; $B$ könnte z.~B. ein Konzept sein, welches die Person $A$ zu einem bestimmten Zeitpunkt repräsentiert.

\subsection{Ereignisse und Termine}%
\label{sec:appendix:queries:events}

\inputminted{cypher}{data/evaluation/personTimeAction.cql}
Diese Anfrage ermittelt die Liste der $label$ aller $relation$-Konzepte und ihrem Patiens, deren Agens die Person ist, die durch \textit{Duane Seppi} bezeichnet wird.
Sofern bekannt, wird den Relationskonzepten der Zeitpunkt zugeordnet, deren Patiens sie sind.

\subsection{Positive und negative Aussagen}%
\label{sec:appendix:queries:neg}

\inputminted{cypher}{data/evaluation/personNegationAction.cql}
