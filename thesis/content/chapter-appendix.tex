% !TEX root = ../main.tex
% chktex-file 46

\chapter{Anhang}%
\label{sec:appendix}

\section{Verwendete Testnachrichten}%
\label{sec:appendix:msgs}

Die Testnachrichten 1--7 stammen aus dem Thread \texttt{bin1/363227} des Enron Threads Corpus~\cite{EnronThreads}.
Die Testnachricht 8 wurde nachträglich hinzugefügt, da in den vorherigen Nachrichten keine Negation vorkam.
Da das implementierte Wissensgraph-Konstruktionsverfahren nicht speziell für E-Mails ausgelegt ist, wird nicht zwischen Absender und Absender-Adresse unterschieden.
Die Absender- und Empfänger-Adressen in den Rohdaten wurden daher durch die Namen der Absender bzw. Empfänger ersetzt.
Abgesehen von dieser Änderung wurden die Inhalte und Absendezeitpunkte der Mails unverändert übernommen.
\inputminted{clojure}{data/evaluation/testdata.clj}

\section{Verwendete Testanfragen}%
\label{sec:appendix:queries}

Zur Evaluation des aus den Testnachrichten konstruierten Wissensgraphen wurden drei Testanfragen verwendet, die jeweils verschiedene Aspekte des Graphen betrachten.
Die Anfragen sind in der Sprache Cypher des Neo4j-Graphdatenbanksystems~\cite{Neo4j} geschrieben.
Die Tests wurden mit Neo4j~3.2.6 und dem Graph Algorithms Plugin~3.2.2.1~\cite{GraphAlgo} durchgeführt.

\subsection{Personen}%
\label{sec:appendix:queries:people}

\inputminted{cypher}{data/evaluation/people.cql}
Diese Anfrage arbeitet im Wesentlichen in zwei Schritten.
Im ersten Schritt wird der Teilgraph aller Konzepte gebildet, die Instanz von \textit{person} und nicht Instanz von \textit{I} oder \textit{you} sind.
Im zweiten Schritt wird dann die Menge aller über $inst$-Kanten schwach zusammenhängender Komponenten des Teilgraphen ermittelt.
Die $label$ der Konzepte innerhalb einer Zusammenhangskomponente sind die Namen bzw.\ Bezeichner, die dieselbe Person in verschiedenen Nachrichten hat.

Der Grund für diesen Aufbau ist, dass eine Person als ein Konzept mit gewissen charakterisierenden Eigenschaften verstanden werden kann.
Die Existenz von $inst(A, B)$ zwischen zwei Personen-Konzepten $A, B$ bedeutet, dass $B$ die charakterisierenden Eigenschaften von $A$ besitzt und somit beide Konzepte auf die selbe Person verweisen.
Für die Modellierung dieser Idee werden keine Koreferenzkanten verwendet, da zwei Konzepte, die auf dieselbe Person verweisen, dennoch verschieden sein können; $B$ könnte z.~B. ein Konzept sein, welches die Person $A$ zu einem bestimmten Zeitpunkt repräsentiert.

\subsection{Ereignisse und Termine}%
\label{sec:appendix:queries:events}

\inputminted{cypher}{data/evaluation/personTimeAction.cql}
Diese Anfrage ermittelt die Liste der $label$ aller $relation$-Konzepte und ihrem Patiens, deren Agens die Person ist, die durch \textit{Duane Seppi} bezeichnet wird.
Sofern bekannt, wird den Relationskonzepten der Zeitpunkt zugeordnet, deren Patiens sie sind.

\subsection{Personenbezogene Daten}%
\label{sec:appendix:queries:numbers}

\inputminted{cypher}{data/evaluation/numbers.cql}
Diese Anfrage ermittelt zuerst alle Konzepte, die Instanz von \textit{number} sind und zugleich Patiens einer Person sind;
die Patiens-Relation zwischen einer Person und einem anderen Konzept wird benutzt, um eine Zugehörigkeit auszudrücken.
Das Ergebnis ist also die Menge aller Nummern und den Personen, denen sie gehören.
Um zu bestimmen, um was für eine Art von Nummer es sich dabei jeweils handelt, wird anschließend für jede Numme $B$ die Menge aller Konzepte $A$ bestimmt, für die $inst(A, B)$ gilt.

\subsection{Positive und negative Aussagen}%
\label{sec:appendix:queries:neg}

\inputminted{cypher}{data/evaluation/personNegationAction.cql}
Diese Anfrage ermittelt zuerst alle Relationen, die von einem Kontext umgeben sind, der den Inhalt einer Nachricht beschreibt, die von \textit{John Doe} gesendet wurde.
Anschließend wird für jede Relation geprüft, ob sie negative umgebende Kontexte hat.
Da bereits in der NLP-Phase doppelte Verneinungen vereinfacht werden, müssen diese nicht beachtet werden.
Es ist daher zwischen drei Fällen zu unterscheiden:
\begin{enumerate}
	\item \textbf{\texttt{POSITIVE}:}
		Die Relation hat keine negativen umgebenden Kontexte und ist somit selbst positiv.
	\item \textbf{\texttt{NEGATIVE}:}
		Die Relation hat einen negativen umgebenden Kontext, deren einziges Kind sie ist.
		In diesem Fall ist die Relation negiert.
	\item \textbf{\texttt{CONDITIONAL}:}
		Die Relation hat mehrere negative umgebende Kontexte oder ist nicht einziges Kind ihres negativen umgebenden Kontextes.
		In diesem Fall ist die Existenz der Relation abhängig von der Existenz anderer Konzepte.
		Prinzipiell ließen sich diese Abhängigkeiten abfragen, der Einfachheit halber, wird dies hier jedoch nicht getan.
\end{enumerate}
Diese Anfrage ist ein gutes Beispiel dafür, dass mit einer Graph-Anfragesprache, in diesem Fall Cypher, Kontexte nur recht umständlich berücksichtigt werden können.
Das Entwickeln eines speziell für Konzeptgraphen ausgelegten Anfragesystems ist daher empfehlenswert, um die Syntax komplexerer Anfragen verständlich zu halten.

\section{Laufzeitergebnisse}%
\label{sec:appendix:perf}

Es folgt eine Tabelle mit den gemessenen Laufzeitergebnissen.
Für jede Nachricht $m_i$ werden die folgenden Daten angegeben:
\begin{enumerate}[noitemsep]
	\item Die Einfügedauer $t_i$ in Sekunden.
	\item Die Veränderung der Einfügedauer $\Delta t_i = t_i - t_{i - 1}$.
	\item Die Anzahl $|\mathbb{C}_i|$ geschlossener Grundatome beim Einfügen von $m_i$.
	\item Die Anzahl neuer Konzepte $c_i$ in $m_i$, d.~h.\ Konzepte mit einem $label$, welches in den vorigen Nachrichten noch nicht vorgekommen ist.
	\item Der Korrelationskoeffizient $Kor_i(t, \mathbb{C}) = Kor({(t_j)}_{j = i, \dots, 57}, {(|\mathbb{C}_j|)}_{j = i, \dots, 57})$.
	\item Der Korrelationskoeffizient $Kor_i(\Delta t, c) = Kor({(\Delta t_j)}_{j = i, \dots, 57}, {(c_j)}_{j = i, \dots, 57})$.
\end{enumerate}

Die Nachrichten $m_1, \dots, m_{24}$ stammen aus \texttt{bin1/300249}, $m_{25}, \dots, m_{32}$ sind die für die Qualitätsuntersuchung verwendeten Nachrichten aus \texttt{bin1/363227}~\tref{sec:appendix:msgs}, $m_{33}, \dots, m_{57}$ stammen aus \texttt{bin5/350143}.
\csvreader[
	longtable={rrrrrS[group-digits=false]S[group-digits=false]},
	separator=comma,
	table head={\toprule$i$ & $t_i$ & $\Delta t_i$ & $|\mathbb{C}_i|$ & $c_i$ & {$Kor_{i}(t, \mathbb{C})$} & {$Kor_{i}(\Delta t, c)$}\\\midrule\endhead\bottomrule\endfoot}, % chktex 21
	late after line=\\,
	head to column names
]{data/evaluation/perfBig.csv}{}{\i& $\time$& $\dtime$& $\input$& $\concepts$& \icorrel& \ccorrel}
