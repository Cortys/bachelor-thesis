% !TEX root = ../main.tex
% chktex-file 46
\chapter{Theoretische Grundlagen}%
\label{sec:theory}

In Kapitel~\ref{sec:related} wurde ein Überblick über das Problemumfeld der Wissensgraphkonstruktion gegeben.
Diese Arbeit baut insbesondere auf den bereits kurz vorgestellten Konzeptgraphen, Stanfords~CoreNLP Bibliothek und der PSL auf.
Für die folgenden Kapitel ist ein Grundverständnis dieser drei Themen notwendig.
Sie werden daher in den folgenden Abschnitten näher beschrieben.

\section{Wissensmodellierung mit Konzeptgraphen}%
\label{sec:theory:cg}

John F. Sowas Konzeptgraphen bilden die Basis der Graphontologie dieser Arbeit.
Wie in~\ref{sec:related:kr:history} beschrieben, sind sie ein auf Existenzgraphen basierendes prädikatenlogisches Kalkül.
Da Sowas eigene Beschreibungen teils etwas unklar sind, werden im folgenden die sog.~\textit{Conceptual Graphs with Cuts}~\cite{Dau2003} vorgestellt.
Sie sind eine formal spezifizierte Teilmenge der Konzeptgraphen, deren Vollständigkeit und Korrektheit bewiesen ist.

\section{Stanford CoreNLP}%
\label{sec:theory:nlp}

\section{Modellierung von Hinge-Loss-MRFs mit PSL}%
\label{sec:theory:psl}
