% !TEX root = ../main.tex
%
\chapter{Einleitung}
\label{sec:intro}

\cleanchapterquote{
The actual world cannot be distinguished from a world of imagination by any description.
Hence the need of pronoun and indices, and the more complicated the subject the greater the need of them.
}{Charles Sanders Peirce}{Mathematiker und Philosoph}

\section{Motivation}
\label{sec:intro:motivation}

In den letzten Jahren hat die Repräsentation von Wissensbasen durch Graphen, sog. Wissensgraphen, immer mehr an Bedeutung gewonnen.
Google, Bing und IBM Watson benutzen solche Wissensgraphen z.~B. zum Beantworten von komplexen Suchanfragen.

Die Grundidee dabei ist es, Entitäten durch Knoten und Relationen durch Kanten abzubilden.
Entitäten können konkrete Dinge, wie z.~B. Personen, aber auch abstrakte Konzepte, wie z.~B. historische Epochen, sein.
Relationen beschreiben beliebige Beziehungen zwischen den Entitäten, z.~B. \(person(\text{Da~Vinci}) \xrightarrow{\text{lived~in}} epoch(\text{Renaissance})\).
Die Entität, von der eine solche Relation ausgeht, wird als Subjekt und die Zielentität als Objekt der Relation bezeichnet.

Die Typen von Entitäten bzw.\ Relationen (z.~B. \(person\) bzw. \(lived~in\)) und deren Bedeutung sind dabei i.~d.~R. formal in einer sog.\ Ontologie spezifiziert.
Die Ontologie beschränkt also die Menge gültiger Wissensgraphen, was eine effiziente maschinelle Verarbeitung der im Graph enthaltenen Informationen ermöglicht.

Da Wissensgraphen in zahlreichen Domänen einsetzbar sind, wird deren automatisierte Konstruktion bereits seit Jahren erforscht.
Manuelles Konstruieren und vor allem anschließendes Warten und Aktualisieren von Wissensgraphen, ist aufgrund der abzubildenden Datenmengen nicht praktikabel.
Bei einer maschinellen automatisierten Konstruktion sind insbesondere zwei Anforderungen problematisch:
\begin{enumerate}
	\item Das Verarbeiten von unstrukturierten Eingaben, wie z.~B. natürlichsprachlichen Texten.
	\item Effizientes Eingliedern neuer Informationen in einen bestehenden Wissensgraphen.
		Dieses Eingliedern von Informationen umfasst im Speziellen:
		\begin{itemize}
			\item \textbf{Entity Resolution:}
				Hinzukommende Entitäten, die bereits im Graphen enthalten sind, müssen als Duplikate erkannt werden.
				Dies ist i.~d.~R. nicht trivial, da die selbe Entität durch viele verschiedene, oftmals vom Kontext abhängige, Token repräsentiert werden kann;
				z.~B. \textit{Bob} vs. \textit{Robert} oder \textit{Der Papst} vs. \textit{Franziskus}.
			\item \textbf{Link Prediction:}
				Hinzukommende Entitäten müssen mit bereits bestehenden Entitäten in Relation gesetzt werden.
				Hinzukommende Relationen können zudem benutzt werden um andere Relationen zu inferieren;
				z.~B. \[female(A) \land B \xrightarrow{\text{son~of}} A \implies A \xrightarrow{\text{mother~of}} B\]
		\end{itemize}
\end{enumerate}

Die Kombination dieser beiden Anforderungen ist interessant, da das meiste verfügbare Wissen in natürlichsprachlicher Textform vorliegt und zudem permanent neues Wissen entsteht.
Ein automatisiertes Wissensgraphkonstruktionsverfahren, welches beide Anforderungen berücksichtigt, ist daher in diversen Domänen von Nutzen. Ein Beispiel hierfür ist die Auswertung von Kommunikationsdaten aus E-Mails oder Chat-Nachrichten.

\section{Problemstellung und Ziele der Arbeit}
\label{sec:intro:goals}

\section{Aufbau der Arbeit}
\label{sec:intro:structure}

\textbf{Kapitel~\ref{sec:related}} \\[0.2em]

\textbf{Kapitel~\ref{sec:theory}} \\[0.2em]

\textbf{Kapitel~\ref{sec:text2kg}} \\[0.2em]

\textbf{Kapitel~\ref{sec:evaluation}} \\[0.2em]

\textbf{Kapitel~\ref{sec:conclusion}} \\[0.2em]
