% !TEX root = main.tex
% chktex-file 46

% **************************************************
% Files' Character Encoding
% **************************************************
\PassOptionsToPackage{utf8}{inputenc}
\usepackage{inputenc}
\usepackage[english,ngerman]{babel}

% **************************************************
% Information and Commands for Reuse
% **************************************************
\newcommand{\thesisTitle}{Probabilistische Online-Wissensgraphkonstruktion aus natürlicher Sprache}
\newcommand{\thesisName}{Clemens Damke}
\newcommand{\thesisMatNr}{7011488}
\newcommand{\thesisSubject}{Bachelorarbeit}
\newcommand{\thesisDate}{\today}
\newcommand{\thesisVersion}{RC 1}

\newcommand{\thesisFirstReviewer}{Prof.~Dr.~Eyke Hüllermeier}
\newcommand{\thesisFirstReviewerUniversity}{\protect{Universität Paderborn}}
\newcommand{\thesisFirstReviewerDepartment}{Institut für Informatik}

\newcommand{\thesisSecondReviewer}{Prof.~Dr.~Axel-Cyrille Ngonga Ngomo}
\newcommand{\thesisSecondReviewerUniversity}{\protect{Universität Paderborn}}
\newcommand{\thesisSecondReviewerDepartment}{Institut für Informatik}

\newcommand{\thesisSupervisor}{Dr.~Theodor Lettmann}

\newcommand{\thesisUniversity}{Universität Paderborn}
\newcommand{\thesisUniversityDepartment}{}
\newcommand{\thesisUniversityInstitute}{Institut für Informatik}
\newcommand{\thesisUniversityGroup}{Intelligente Systeme}
\newcommand{\thesisUniversityCity}{Paderborn}
\newcommand{\thesisUniversityStreetAddress}{Pohlweg 51}
\newcommand{\thesisUniversityPostalCode}{33098}


% **************************************************
% Debug LaTeX Information
% **************************************************
%\listfiles


% **************************************************
% Load and Configure Packages
% **************************************************

% Colors:
\usepackage[usenames, dvipsnames, svgnames, table]{xcolor}

\definecolor{blau}{HTML}{355FB3}
\definecolor{rot}{HTML}{B33535}
\definecolor{gruen}{HTML}{3BB335}
\definecolor{dunkelblau}{HTML}{1E3666}
\definecolor{hellblau}{HTML}{8ea7d7}

\usepackage{minted}
\usepackage{etoolbox,xpatch}
\makeatletter
\AtBeginEnvironment{minted}{\dontdofcolorbox}
\def\dontdofcolorbox{\renewcommand\fcolorbox[4][]{##4}}
\xpatchcmd{\inputminted}{\minted@fvset}{\minted@fvset\dontdofcolorbox}{}{}
\makeatother
\setminted{
	fontsize=\footnotesize,
	numbers=left,
	tabsize=4,
	breaklines=true
}

\PassOptionsToPackage{% setup clean thesis style
    figuresep=space,
    sansserif=false,
    hangfigurecaption=false,
    hangsection=true,
    hangsubsection=true,
    colorize=full,
    colortheme=custom,
	colormain=dunkelblau,
	coloraccessory=blau,
    bibsys=bibtex,
    bibfile=bib-refs,
    bibstyle=alphabetic,
    wrapfooter=false,
}{cleanthesis}
\usepackage{cleanthesis}

\usepackage{mathtools}
\newcommand\numberthis{\addtocounter{equation}{1}\tag{\theequation}}

\usepackage{graphicx}
\usepackage{tikz}
\usetikzlibrary{arrows,positioning}
\usetikzlibrary{calc}
\newcommand{\tikzmark}[1]{\tikz[overlay,remember picture] \node (#1) {};} % chktex 1

\usepackage{pgfplots}
\usepackage{pgfplotstable}
\pgfplotsset{compat=1.14}
\usepgfplotslibrary{dateplot, statistics}
\pgfplotsset{
    cycle list={blau\\rot\\gruen\\},
}

\usepackage{listings}
\lstset{basicstyle=\ttfamily,breaklines=true}

\usepackage{tasks}
\settasks{counter-format=tsk[1].}

\usepackage[bguq]{frege}
\usepackage{stmaryrd}
\usepackage{multicol}
\usepackage{pbox}
\usepackage{longtable}
\usepackage{booktabs}
\usepackage{csvsimple}
\usepackage{siunitx}

\hypersetup{% setup the hyperref-package options
    pdftitle={\thesisTitle},    %   - title (PDF meta)
    pdfsubject={\thesisSubject},%   - subject (PDF meta)
    pdfauthor={\thesisName},    %   - author (PDF meta)
    plainpages=false,           %   -
    colorlinks=false,           %   - colorize links?
    pdfborder={0 0 0},          %   -
    breaklinks=true,            %   - allow line break inside links
    bookmarksnumbered=true,     %
    bookmarksopen=true          %
}

\makeatletter
\newcounter{rulecount}[section]
\newcommand{\ruleno}[1]{\ensuremath{r_{#1}}}
\newcommand{\rulecurrent}{\ruleno{\therulecount}}
\newcommand{\rulemark}[1]{\refstepcounter{rulecount}(\rulecurrent)\ltx@label{#1}} % chktex 36
\newcommand{\ruleref}[1]{\ruleno{\ref{#1}}}
\makeatother
