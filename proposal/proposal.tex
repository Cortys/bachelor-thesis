\documentclass[11pt, a4paper]{scrreprt}

\usepackage[ngerman]{babel}
\usepackage[utf8]{inputenc}
\usepackage[T1]{fontenc}
\usepackage[light]{roboto}
\usepackage[all]{nowidow}
\usepackage{graphicx, curves, float, rotating}
\usepackage[usenames, dvipsnames, svgnames, table]{xcolor}
\usepackage{mathtools}

% Page margins:
\usepackage[
	top=25mm,
	bottom=30mm,
	left=25mm,
	right=25mm
]{geometry}

% Colors:
\definecolor{blau}{HTML}{355FB3}
\definecolor{rot}{HTML}{B33535}
\definecolor{gruen}{HTML}{3BB335}

% Fonts:
\renewcommand{\rmdefault}{ppl}
\addtokomafont{subject}{\sffamily\mdseries}
\addtokomafont{title}{\usekomafont{subject}\color{blau}}
\addtokomafont{author}{\usekomafont{subject}}
\addtokomafont{date}{\usekomafont{subject}\normalsize}
\addtokomafont{publishers}{\usekomafont{subject}}
\addtokomafont{titlehead}{\raggedleft\sffamily\mdseries\small\textit}

\addtokomafont{pagenumber}{\sffamily\mdseries}

\addtokomafont{section}{\usekomafont{title}}

% Remove chapter numbers from section numbering:
\renewcommand*\thesection{\arabic{section}}

% Disable paragraph indent:
\setlength{\parindent}{0pt}

\begin{document}

\frontmatter
\titlehead{Entwurf 1}
\subject{Bachelorarbeit Proposal}
\title{
	Probabilistische online\\
	Wissensgraphkonstruktion\\
	aus natürlicher Sprache
}
\author{
	Clemens Damke\\[1ex]
	Matrikelnr. 7011488
}
\publishers{
	{\normalsize betreut von}\\[2ex]
	Prof.~Dr.~Eyke Hüllermeier\\
	Intelligente Systeme\\
	Institut für Informatik\\
	Universität Paderborn
}
\maketitle

\section{Motivation und Hintergrund}

In den letzten Jahren hat die Repräsentation von Wissensbasen durch Graphen immer mehr an Bedeutung gewonnen.
Google benutzt solche sog.\ Wissensgraphen z.~B. zum Beantworten von komplexen Suchanfragen.\\

Die Grundidee dabei ist, Entitäten durch Knoten und Relationen durch Kanten abzubilden.
Entitäten können konkrete Dinge, wie z.~B. Personen, aber auch abstrakte Konzepte, wie z.~B. historische Epochen, sein.
Relationen beschreiben beliebige Beziehungen zwischen den Entitäten, z.~B. $Person(\text{Da~Vinci}) \xrightarrow{\text{lebte~in}} Epoche(\text{Renaissance})$.\\

Da solche Graphen in zahlreichen Domänen einsetzbar sind, wird deren automatisierte Konstruktion bereits seit Jahren erforscht.
Manuelles Konstruieren und vor allem anschließendes Warten und Aktualisieren von Wissensgraphen ist aufgrund der abzubildenden Datenmengen nicht praktikabel.
Bei einer maschinellen automatisierten Konstruktion sind insbesondere zwei Anforderungen problematisch:
\begin{enumerate}
	\item Das Verarbeiten von unstrukturierten Eingaben, wie z.~B. natürlichsprachlichen Texten.
	\item Effizientes Eingliedern neuer Informationen in einen bestehenden Wissensgraphen.
		Dieses Eingliedern von Informationen umfasst im Speziellen:
		\begin{enumerate}
			\item \textbf{Entity Resolution:}
				Hinzukommende Entitäten, die bereits im Graphen enthalten sind, müssen als Duplikate erkannt werden.
				Dies ist i.~d.~R. nicht trivial, da die selbe Entität durch viele verschiedene, oftmals vom Kontext abhängige, Token repräsentiert werden kann;
				z.~B. \textit{Bob} vs. \textit{Robert} oder \textit{Der Papst} vs. \textit{Franziskus}.
			\item \textbf{Link Prediction:}
				Hinzukommende Entitäten müssen mit bereits bestehenden Entitäten in Relation gesetzt werden.
				Hinzukommende Relationen können zudem benutzt werden um andere Relationen zu inferieren;
				z.~B. $$Weiblich(A) \land B \xrightarrow{\text{Sohn~von}} A \implies A \xrightarrow{\text{Mutter~von}} B$$
		\end{enumerate}
\end{enumerate}

Die Kombination dieser beiden Anforderungen ist interessant, da das meiste verfügbare Wissen in natürlichsprachlicher Textform vorliegt und zudem permanent neues Wissen entsteht.
Ein automatisiertes Wissensgraphkonstruktionsverfahren sollte daher beide Anforderungen berücksichtigen.\\

Neben diesen Anforderungen bzgl.\ der Extraktion von Wissen ist zudem wichtig, wie genau der Graph repräsentiert wird.
Zusätzlich zu Knoten bzw.\ Entitäten und Kanten bzw.\ Relationen sind oftmals weitere Metadaten relevant.
Dazu zählt insbesondere die Inferernzkonfidenz des Link Predictors.
Da natürlichsprachliche Eingabeinformationen häufig unvollständig oder fehlerhaft sind, ist es für die Interpretation und Analyse des resultierenden Graphen hilfreich jeder Relation eine Konfidenz $\in [0, 1]$ zuzuordnen.
Das Ergebnis ist ein sog.\ probabilistischer Wissensgraph.

\section{Ziele der Arbeit}

\section{Verwandte Arbeiten}

\section{Inhalte}

\section{Zeitplan}

\end{document}
