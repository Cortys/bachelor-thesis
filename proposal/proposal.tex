\documentclass[11pt, a4paper]{scrreprt}

\usepackage[ngerman]{babel}
\usepackage[utf8]{inputenc}
\usepackage[T1]{fontenc}
\usepackage[light]{roboto}
\usepackage[all]{nowidow}
\usepackage{graphicx, curves, float, rotating}
\usepackage[usenames, dvipsnames, svgnames, table]{xcolor}
\usepackage{mathtools}

% Page margins:
\usepackage[
	top=25mm,
	bottom=30mm,
	left=25mm,
	right=25mm
]{geometry}

% Colors:
\definecolor{blau}{HTML}{355FB3}
\definecolor{rot}{HTML}{B33535}
\definecolor{gruen}{HTML}{3BB335}

% Fonts:
\renewcommand{\rmdefault}{ppl}
\addtokomafont{subject}{\sffamily\mdseries}
\addtokomafont{title}{\usekomafont{subject}\color{blau}}
\addtokomafont{author}{\usekomafont{subject}}
\addtokomafont{date}{\usekomafont{subject}\normalsize}
\addtokomafont{publishers}{\usekomafont{subject}}
\addtokomafont{titlehead}{\raggedleft\sffamily\mdseries\small\textit}

\addtokomafont{pagenumber}{\sffamily\mdseries}

\addtokomafont{section}{\usekomafont{title}}

% Remove chapter numbers from section numbering:
\renewcommand*\thesection{\arabic{section}}

\begin{document}

\frontmatter
\titlehead{Entwurf 1}
\subject{Bachelorarbeit Proposal}
\title{
	Online Wissensgraphkonstruktion\\
	aus natürlicher Sprache
}
\author{
	Clemens Damke\\[1ex]
	Matrikelnr. 7011488
}
\publishers{
	{\normalsize betreut von}\\[2ex]
	Prof.~Dr.~Eyke Hüllermeier\\
	Intelligente Systeme\\
	Institut für Informatik\\
	Universität Paderborn
}
\maketitle

\section{Motivation und Hintergrund}

In den letzten Jahren hat die Repräsentation von Wissensbasen durch Graphen immer mehr an Bedeutung gewonnen.
Google benutzt solche Wissensgraphen z.~B. zum Beantworten von komplexen Suchanfragen.\\

Die Grundidee dabei ist Entitäten durch Knoten und Relationen durch Kanten abzubilden.
Entitäten können konkrete Dinge, wie z.~B. Personen, aber auch abstrakte Konzepte, wie z.~B. historische Epochen, sein.
Relationen beschreiben beliebige Beziehungen zwischen den Entitäten, z.~B. $Person(\text{Da Vinci}) \xrightarrow{\text{lebte in}} Epoche(\text{Renaissance})$.\\

Zur Konstruktion dieser Graphen existieren verschiedene Verfahren.
Viele davon erfordern allerdings entweder eine einheitlich strukturierte Eingabe oder menschliches Feedback zur Klassifikation und Verifikation von Entitäten und Relationen.

\section{Ziele}

\section{Verwandte Arbeiten}

\section{Inhalte}

\section{Zeitplan}

\end{document}
